%%% Для сборки выполнить 2 раза команду: pdflatex <имя файла>

\documentclass[a4paper,12pt]{article}

\usepackage{ucs}
\usepackage[utf8x]{inputenc}
\usepackage[russian]{babel}
%\usepackage{cmlgc}
\usepackage{graphicx}
\usepackage{listings}
\usepackage{xcolor}
\usepackage{titlesec}
%\usepackage{courier}

\makeatletter
\renewcommand\@biblabel[1]{#1.}
\makeatother

\newcommand{\myrule}[1]{\rule{#1}{0.4pt}}
\newcommand{\sign}[2][~]{{\small\myrule{#2}\\[-0.7em]\makebox[#2]{\it #1}}}

% Поля
\usepackage[top=20mm, left=30mm, right=10mm, bottom=20mm, nohead]{geometry}
\usepackage{indentfirst}

% Межстрочный интервал
\renewcommand{\baselinestretch}{1.50}

% ------------------------------------------------------------------------------
% minted
% ------------------------------------------------------------------------------
\usepackage{minted}


% ------------------------------------------------------------------------------
% tcolorbox / tcblisting
% ------------------------------------------------------------------------------
\usepackage{xcolor}
\definecolor{codecolor}{HTML}{FFC300}

\usepackage{tcolorbox}
\tcbuselibrary{most,listingsutf8,minted}

\tcbset{tcbox width=auto,left=1mm,top=1mm,bottom=1mm,
right=1mm,boxsep=1mm,middle=1pt}

\newtcblisting{myr}[1]{colback=codecolor!5,colframe=codecolor!80!black,listing only, 
minted options={numbers=left, style=tcblatex,fontsize=\tiny,breaklines,autogobble,linenos,numbersep=3mm},
left=5mm,enhanced,
title=#1, fonttitle=\bfseries,
listing engine=minted,minted language=r}

%%%%%%%%%%%%%%%%%%%%%%%%%%%%%%%%%%%%%%%

\begin{document}

%%%%%%%%%%%%%%%%%%%%%%%%%%%%%%%
%%%                         %%%
%%% Начало титульного листа %%%

\thispagestyle{empty}
\begin{center}


\renewcommand{\baselinestretch}{1}
{\large
{\sc Петрозаводский государственный университет\\
Институт математики и информационных технологий\\
Кафедра информатики и математического обеспечения
}
}

\end{center}


\begin{center}
%%%%%%%%%%%%%%%%%%%%%%%%%
%
% Раскомментируйте (уберите знак процента в начале строки)
% для одной из строк типа направления  - бакалавриат/
% магистратура и для одной из
% строк Вашего направление подготовки
%
 Направление подготовки бакалавриата \\
% 01.03.02 --- Прикладная математика и информатика \\
% 09.03.02 --- Информационные системы и технологии \\
09.03.04 --- Программная инженерия \\
%%%%%%%%%%%%%%%%%%%%%%%%%
\end{center}

\vfill

\begin{center}
{\normalsize 
	Отчет по практике}

\medskip

%%% Название работы %%%
	{\Large \sc {Разработка приложения \\ <<Курсы Валют>> }} \\
\end{center}

\medskip

\begin{flushright}
\parbox{11cm}{%
\renewcommand{\baselinestretch}{1.2}
\normalsize
	Выполнил:\\
% Выполнили:\\
%%% ФИО студента %%%
студента 1 курса группы 22107
\begin{flushright}
	Н. Д. Беленков \sign[подпись]{4cm}
\end{flushright}

%%% Второй участник %%%
% студента 1 курса группы 2210X
% \begin{flushright}
% 	И. О. Фамилия \sign[подпись]{4cm}
% \end{flushright}

%%%%%%%%%%%%%%%%%%%%%%%%%
% девушкам применять "Выполнила" и "студентка"
%%%%%%%%%%%%%%%%%%%%%%%%%
}
\end{flushright}

\vfill

\begin{center}
\large
    Петрозаводск --- 2021
\end{center}

%%% Конец титульного листа  %%%
%%%                         %%%
%%%%%%%%%%%%%%%%%%%%%%%%%%%%%%%

%%%%%%%%%%%%%%%%%%%%%%%%%%%%%%%%
%%%                          %%%
%%% Содержание               %%%

\newpage

\tableofcontents

%%% Содержание              %%%
%%%                         %%%
%%%%%%%%%%%%%%%%%%%%%%%%%%%%%%%

%%%%%%%%%%%%%%%%%%%%%%%%%%%%%%%%
%%%                          %%%
%%% Введение                 %%%

%%% В введении Вы должны описать предметную область, с которой связана %%%
%%% Ваша работа, показать её актуальность, вкратце определить цель     %%%
%%% разработки					       %%%


\newpage
\section*{Введение}
\addcontentsline{toc}{section}{Введение}

Цель проекта: Разработать приложение для вывода курсов валют на запрошенную пользователем дату.

Задачи проекта: 
%%% Пример создания списков %%%
\begin{enumerate} 
    \item Разработать функцию для скачивания файла из Интернета.
    \item Разработать модуль для парсинга полученного xml.
    \item Разработать графический интерфейс пользователя.
    \item Реализовать приложение при помощи разработанных модулей и QtWidgets.
\end{enumerate}

В наше время связь между странами стала намного более сильной. Сейчас намного проще получить из-за границы вещь, которую не купить в вашей стране, однако отслеживать цены таким образом сложнее. Курсы валют меняются каждый день, и иногда нам необходимо знать, сколько стоил доллар США или японская иена неделю назад. Таким образом, целью данного проекта стала разработка программы, способной вывести курс валют на заданную дату согласно Центральному Банку Российской Федерации.

%%% Пример добавления изображения %%%
%%% называйте изображение латиницей %%%
% \begin{figure}[h]
%   \centering
%   \includegraphics[width=0.8\textwidth]{images/lisp-eng.png}
%   \caption{\label{fig:trans-vs-reactive}Спойлер для программных инженеров}
% \end{figure}

%%%                          %%%
%%%%%%%%%%%%%%%%%%%%%%%%%%%%%%%%

%%%%%%%%%%%%%%%%%%%%%%%%%%%%%%%
%%% Требования к приложению %%%
\newpage
\section{Требования к приложению}
% \subsection{Подраздел}

С точки зрения пользователя приложение должно иметь следующие функции :
\begin{itemize}
    \item Выбор необходимой даты.
    \item Вывод курсов валют на выбранную дату.
\end{itemize}
 
%%%                                     %%%
%%%%%%%%%%%%%%%%%%%%%%%%%%%%%%%%%%%%%%%%%%%

%%%%%%%%%%%%%%%%%%%%%%%%%%%%%%%%%%%%%%%%%%%
%%%                                     %%%
%%% Проектирование приложения           %%%
\newpage
\section{Проектирование приложения}

Приложение состоит из нескольких модулей : 
\begin{enumerate}
    \item main.cpp - главный модуль, вызывающий графических интерфейс.
    \item mainwindow.cpp - модуль главного окна. Включает следующие функции :
    \begin{itemize}
        \item get\_url() - отправляет запрос для получения xml на запрошенную дату.
    \end{itemize}
    A также следующие слоты:
    \begin{itemize}
        \item replyFinished() - вызывает обработчик xml, когда приходит ответ на запрос от get\_url.
        \item on\_dateEdit\_dateChanged() - вызывает функцию get\_url, когда пользователь изменяет выбранную дату.
    \end{itemize}
    \item helper.cpp - модуль для парсинга xml. Включает следующие функции :
    \begin{itemize}
        \item parseXml() - разбор xml на отдельные валюты.
        \item parseOneItem() - разбор свойств каждой валюты и объединение их в строку.
    \end{itemize}
    \item mainwindow.ui - модуль, определяющий внешний вид графического интерфейса приложения.
\end{enumerate}

%%%                          %%%
%%%%%%%%%%%%%%%%%%%%%%%%%%%%%%%%

%%%%%%%%%%%%%%%%%%%%%%%%%%%%%%%%
%%%                          %%%
%%% Реализация приложения    %%%
\newpage
\section{Реализация приложения}

Для реализации приложения были использованы язык C++, а также модули библиотеки Qt : QtCore, QtGui, QtWidgets и QtNetwork. \\
\begin{itemize}
    \item Количество модулей : 4
    \item Количество функций : 5
    \item Количество C++ классов : 2
    \item Количество Qt слотов : 2
    \item Количество строк C++ кода : 124
\end{itemize}


%%% Если необходимо вставками оформляются исключительно небольшие фрагменты кода.
%%% Для больших фрагментов используте приложение (пример после заключения)


% Первый вариант оформления кода:

% \begin{minted}[linenos, fontsize=\footnotesize, frame=lines]
% {c}
% #include <stdio.h>
% void main()
% {
%     printf("Hello world\n");
% }
% \end{minted}

% Второй вариант оформление кода:
% \begin{verbatim}
% #include <stdio.h>
% void main()
% {
%     printf("Hello world\n");
% }
% \end{verbatim}

%%%                          %%%
%%%%%%%%%%%%%%%%%%%%%%%%%%%%%%%%

%%%%%%%%%%%%%%%%%%%%%%%%%%%%%%%%
%%%                          %%%
%%% Заключение               %%%

\newpage
\section*{Заключение}
\addcontentsline{toc}{section}{Заключение}
В результате нам удалось разработать приложение, которое может вывести курсы валют на запрошенную пользователем дату, начиная с 1992 года и до наших дней. Программа имеет интуитивно понятный интерфейс, для взаимодействия с которым пользователю нужно всего лишь ввести интересующий его день, а программа ыведет курсы валют на этот день (или на день,в который курс обновлялся до этого). Приложение получает xml с официальных ресурсамов ЦБ РФ и выводит ответ пользователю в виде строк текста.
\parВ ходе данной работы я получил опыт работы с QtNetwork - модулем библиотеки Qt,а также с методами разбора xml файлов, опираясь на функции данной библиотеки.

%%%                          %%%
%%%%%%%%%%%%%%%%%%%%%%%%%%%%%%%%

%%%%%%%%%%%%%%%%%%%%%%%%%%%%%%%%
%%%                          %%%
%%% Приложение               %%%

% \newpage
% \appendix
%\section*{Приложение}
%\addcontentsline{toc}{section}{Приложение}
%\titleformat{\section}[display]
%  {\normalfont\Large\bfseries}
%  {Приложение\ \thesection}
%  {0pt}{\Large\centering}
%\renewcommand{\thesection}{\Asbuk{section}}

% \section*{Приложение А. Hello World}
% \addcontentsline{toc}{section}{Приложение А. Hello World}
% \begin{minted}[linenos, fontsize=\footnotesize, frame=lines]
% {c}
% #include <stdio.h>
% #include <string.h>

% int main()
% {
%   const char * msg = "Hello world!\n";
%   int printf_res = printf(msg);
%   if (printf_res < strlen(msg))
%   {
%     return 1;
%   } else {
%     return 0;
%   }
% }
% \end{minted}

%%% Ещё одно приложение
% \newpage
% \section*{Приложение Б.}
% \addcontentsline{toc}{section}{Приложение Б.}

%%%                          %%%
%%%%%%%%%%%%%%%%%%%%%%%%%%%%%%%%
\end{document}
